\documentclass{beamer}
\usetheme{Madrid}
\usepackage{amsmath}
\usepackage{amsfonts}
\usepackage{graphicx}
\usepackage{booktabs}
\title{RMST-Based Sample Size Estimation}
\subtitle{Methods in survRM2, nphRCT, and gsDesign2}
\author{Arnab Aich}
\institute{UTHSC-PM}

\begin{document}

\frame{\titlepage}

\begin{frame}{Overview}
\begin{itemize}
    \item Restricted Mean Survival Time (RMST) is a robust, model-free summary measure for time-to-event data: \textcolor{red}{change $\tau$ to L to differentiate with end of study}
    \[
    \mu(\tau) = \mathbb{E}[\min(T, \tau)] = \int_0^\tau S(t) \, dt
    \]
    \item Unlike the hazard ratio, RMST does not assume proportional hazards. \textcolor{red}{Conventional modeling of the hazard rate assumes proportional hazards. If RMST is calculated from conventional PH models, then your statement is not true.}
    \item Used increasingly for both inference and design in clinical trials.
    \item This talk focuses on three R packages that support RMST-based estimation or sample size:
    \begin{itemize}
        \item \texttt{survRM2}
        \item \texttt{nphRCT}
        \item \texttt{gsDesign2}
    \end{itemize}
\end{itemize}
\end{frame}

% Slide 2 - survRM2
\begin{frame}{survRM2::rmst2 - Estimation}
\begin{itemize}
    \item \texttt{rmst2} estimates RMST under a treatment-control setting:
    \[
    \hat{\mu}_g(\tau) = \int_0^\tau \hat{S}_g(t) \, dt
    \]
    where $\hat{S}_g(t)$ is the Kaplan-Meier estimator in group $g$.
    
    \item RMST difference: 
    \[
    \hat{\Delta}_{\text{RMST}} = \hat{\mu}_1(\tau) - \hat{\mu}_0(\tau)
    \]
    
    \item Standard error is computed via Greenwood's formula and delta method.
    
    \item Wald-type test:
    \[
    T = \frac{\hat{\Delta}_{\text{RMST}}}{\text{SE}(\hat{\Delta})}
    \]
    
    \item Though no built-in sample size calculator, RMST difference and SE from \texttt{survRM2} can be used as inputs for power/sample size calculations.
\end{itemize}
\end{frame}

% Slide 4 - nphRCT
\begin{frame}{nphRCT - Simulation-Based Estimation}
\begin{itemize}
    \item \texttt{nphRCT} supports flexible trial designs including non-proportional hazards and delayed treatment effects.

    \item RMST-based power is computed using Monte Carlo simulation:
    \begin{enumerate}
        \item Simulate survival data under control and treatment groups.
        \item Estimate $\hat{\mu}_1(\tau)$ and $\hat{\mu}_0(\tau)$ using the Kaplan-Meier method.
        \item Compute RMST difference $\hat{\Delta}_{\text{RMST}}$.
        \item Evaluate proportion of simulations where:
        \[
        \hat{\Delta}_{\text{RMST}} > z_{1-\alpha} \cdot \text{SE}
        \]
        to estimate power.
    \end{enumerate}

    \item Sample size is adjusted iteratively to achieve desired power.
    \item Allows incorporating cure models, lagged effects, and complex hazard structures.
\end{itemize}
\end{frame}

% Slide 7 - gsDesign2
\begin{frame}{gsDesign2 - RMST-Based Design}
\begin{itemize}
    \item \texttt{gsDesign2} extends group sequential methods to nonparametric endpoints like RMST. \textcolor{red}{This package is not as important because it focuses on group sequential design.}

    \item Test statistic:
    \[
    T = \frac{\hat{\mu}_1(\tau) - \hat{\mu}_0(\tau)}{\sqrt{\hat{\sigma}^2 / n}}
    \]
    where $\hat{\sigma}^2$ is the estimated variance of the RMST difference.

    \item Procedure:
    \begin{enumerate}
        \item Specify $\tau$, effect size ($\Delta$), and desired power.
        \item Estimate variance $\hat{\sigma}^2$ from historical data or \texttt{survRM2}.
        \item Solve for sample size using:
        \[
        n = \left( \frac{Z_{1-\alpha/2} + Z_{1-\beta}}{\Delta / \sigma} \right)^2
        \]
    \end{enumerate}

    \item Allows interim analysis, adaptive design, and stopping rules under RMST framework.
\end{itemize}
\end{frame}

\begin{frame}{Conclusion}
\begin{itemize}
    \item RMST is a powerful alternative to hazard ratios, especially under non-proportional hazards.
    \item \texttt{survRM2} provides robust estimation and hypothesis testing.
    \item \texttt{nphRCT} and \texttt{gsDesign2} extend this to simulation and adaptive design for planning.
\end{itemize}
\end{frame}

\end{document}


